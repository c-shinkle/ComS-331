\documentclass{article}

\usepackage{amsmath}
\usepackage{amssymb}
\usepackage{graphicx}
\usepackage{epsfig}

% epsf figures (inline or centered)
\newcommand{\FIG}[1]{\begin{center}
  \mbox{\epsfclipoff\epsffile{#1.eps}}
  \end{center}}
\def\epsfsize#1#2{.7#1}	      % change 1.0 to any zoom factor you want

% numeric sets
\newcommand{\Reals}{\mathbb{R}}      % real numbers
\newcommand{\Naturals}{\mathbb{N}}   % natural numbers
\newcommand{\Integers}{\mathbb{Z}}   % integer numbers
\newcommand{\Rationals}{\mathbb{Q}}  % rational numbers
\newcommand{\Complexes}{\mathbb{C}}  % complex numbers
 
 \begin{document}
 	\title{COM S 331\large \\Homework 2}
	\author{Christian Shinkle}
	\maketitle
	\begin{enumerate}
		\item DFA:
		\FIG{hw2DFA1}
		\item In plain English, any binary string that's at least 4 bits long and has a 0 for the fourth to last bit. The regular expression for it is: \[(0+1)^*0(0+1)(0+1)(0+1)\] \newpage The NFA that defines this DFA is: \FIG{hw2NFA1}
		\item This is a DFA:\def\epsfsize#1#2{.6#1} \FIG{hw2DFA2}
	\end{enumerate}
 \end{document}