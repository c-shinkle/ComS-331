\documentclass[12pt]{article}
\usepackage{epsfig}
\listfiles

\title{\bf How to include ``.eps'' files in LaTeX}
\author{Gianfranco Ciardo}

\setlength{\oddsidemargin}{-0.25in}
\setlength{\topmargin}{-0.5in}
\setlength{\headheight}{0cm}
\setlength{\headsep}{0cm}
\setlength{\textheight}{10in}
\setlength{\textwidth}{7in}
\setlength{\topskip}{0cm}
\renewcommand{\baselinestretch}{1.1}
\renewcommand{\arraystretch}{1.2}
\renewcommand{\dbltopfraction}{0.9}
\renewcommand{\topfraction}{0.9}
\renewcommand{\bottomfraction}{0.9}
\renewcommand{\dblfloatpagefraction}{0.99}
\renewcommand{\floatpagefraction}{0.99}
\renewcommand{\textfraction}{0.01}

% epsf figures (inline or centered)
\newcommand{\INLINEFIG}[1]{\epsfclipoff\epsffile{#1.eps}}
\newcommand{\FIG}[1]{\begin{center}
  \mbox{\epsfclipoff\epsffile{#1.eps}}
  \end{center}}
\def\epsfsize#1#2{1.0#1}	      % change 1.0 to any zoom factor you want

\pagestyle{empty}

\date{}

\begin{document}

\maketitle

\thispagestyle{empty}

First, you need to generate a ``.eps'' (encapsulated postscript) file using
one of the tools available to you.
Then you can include it in a LaTeX file.
To see how, examine the source file for this page.

This is an example of an inline figure, \INLINEFIG{Figure}, while
this is an example of the same figure centered in a space by itself:
\FIG{Figure}

You can scale the figure as you want:
for example, this is the same figure, just twice as large:
{\def\epsfsize#1#2{2.0#1}\FIG{Figure}}
or one fourth as large:
{\def\epsfsize#1#2{0.25#1}\FIG{Figure}}

In multipage documents, it is usually more appropriate to include pictures
only in the ``figure'' environment, as done in Fig.~\ref{FIG:myfig}.

\begin{figure}
\FIG{Figure}
\caption{An example of floating figure.}
\label{FIG:myfig}
\end{figure}

\end{document}
