\documentclass{article}

\usepackage{amsmath}
\usepackage{amssymb}
\usepackage{graphicx}
\usepackage{epsfig}

% epsf figures (inline or centered)
\newcommand{\FIG}[1]{\begin{center}
  \mbox{\epsfclipoff\epsffile{#1.eps}}
  \end{center}}
\def\epsfsize#1#2{1.0#1}	      % change 1.0 to any zoom factor you want

% numeric sets
\newcommand{\Reals}{\mathbb{R}}      % real numbers
\newcommand{\Naturals}{\mathbb{N}}   % natural numbers
\newcommand{\Integers}{\mathbb{Z}}   % integer numbers
\newcommand{\Rationals}{\mathbb{Q}}  % rational numbers
\newcommand{\Complexes}{\mathbb{C}}  % complex numbers
 \newcommand{\ra}{\rightarrow}
 \begin{document}
 	\title{COM S 331\large \\Homework 4}
	\author{Christian Shinkle}
	\maketitle
	\begin{enumerate}
		\item Proof by Contradiction: Assume $L$ is regular, then\\ $\exists m\in \Naturals,\forall x\in L$, $|x|>m, \exists a, b, c \in w^*, x=abc, |ab|\leq m, |b|>0,\\ \forall k\in\Naturals, ab^{k}c\in L$.
		Consider $ww$ and denote each substring $w$ as $w_1$ and $w_2$. Then, decompose it in any possible way into $abc$ satisfying $|ab|\leq m$,	$|b|>0$. Then, set $k=0$ for any $b$ that was chosen. Three cases can arise: 
		\begin{enumerate}
			\item b is entirely in $w_1$
			\item b is entirely in $w_2$
			\item b is partially in $w_1$ and $w_2$
		\end{enumerate}
		The first two cases are trivial because under pumping $b^0$ will make either $w_1$ or $w_2$ no longer equal to the original $w$. If the thrid cases arises, then both $w_1$ and $w_2$ will both not be equal to the $w$. In all three cases, a contradiction has occured.
		\item Proof by Contradiction: Assume $L$ is regular. 
		Then, using the Closure property of Homomorphism, we can say $L^{'}=\{w\in \{0,1\}^*:|w|_0\neq|w|_1\}$ is obtained from the homomorphism $a\ra0$ and $b\ra1$. 
		Then, using the Closure property of Complementation, we say\\ $L^{''}=\{w\in \{0,1\}^*:|w|_0=|w|_1\}$. 
		Then, using the Closure property of Concatenation, we say $L^{''}(M_1)*L^{''}(M_2)=L^{'''}=\{w\in \{0,1\}^*: ww\}$. However, note that $L^{'''}$ is regular, but using the proof from problem one, $L^{'''}$ is nonregular, a contradiction.
	\end{enumerate}
 \end{document}