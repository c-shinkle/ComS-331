\documentclass{article}

\usepackage{amsmath}
\usepackage{amssymb}
\usepackage{graphicx}
\usepackage{epsfig}

% epsf figures (inline or centered)
\newcommand{\FIG}[1]{\begin{center}
  \mbox{\epsfclipoff\epsffile{#1.eps}}
  \end{center}}
\def\epsfsize#1#2{1.0#1}	      % change 1.0 to any zoom factor you want

% numeric sets
\newcommand{\Reals}{\mathbb{R}}      % real numbers
\newcommand{\Naturals}{\mathbb{N}}   % natural numbers
\newcommand{\Integers}{\mathbb{Z}}   % integer numbers
\newcommand{\Rationals}{\mathbb{Q}}  % rational numbers
\newcommand{\Complexes}{\mathbb{C}}  % complex numbers

\begin{document}
\title{COMS 331\large \\
	Homework 0}
\author{Christian Shinkle}
\maketitle
\begin{enumerate}
	\item 
		\fbox{\parbox{\dimexpr\linewidth-2\fboxsep-2\fboxrule\relax} 
			{
			This is an inline equation: $x+y=3$\\
			This is a displayed equation: 
			\[x+\frac{y}{z-\sqrt{3}}=2.\]
			This is how you can define a piece-wise linear function:
			\[f(x)=
				\begin{cases}
				3x+2 & \text{if $x<0$}\\
				7x+2 & \text{if $x\geq 0$ and $x<10$}\\
				5x+22 & \text{otherwise}
				\end{cases} 
			\]
			This is a matrix:
			\begin{center}
				\begin{tabular} {|c|c|c|c|} 
				\hline 
				9 & 9 & 9 & 9 \\
				\hline
				6 & 6 & 6 & \\
				\hline
				3 & & 3 & 3 \\ 
				\hline
				\end{tabular}
			\end{center}
			This is a figure incorporated in a LaTeX file\\
			\FIG{fig}
			}
		}
	\item 
		In order to show that $\Naturals \ $and$\ \Integers$ are equinumerous, we must define a bijection between them. Note the function f : $\Naturals \to \Integers$ where 
		\[f(n)=
			\begin{cases}
			$-(n/2)$ & \text{if $n$ is even}\\
			$(n+1)/2$ & \text{if $n$ is odd }\\
			\end{cases} 
		\]
		First, we prove $f$ is one-to-one. Let $\forall n,m \in \Naturals ,\  f(n) = f(m)$. Then we must prove 2 cases. \\
		Case 1: $n$ is even\\
		\[ f(n) = f(m)\] 
		\[-(n/2) = -(m/2)\]
		\[n/2 = m/2\]
		\[n = m\]
		Case 2: $n$ is odd\\
		\[f(n) = f(m)\]
		\[(n+1)/2 = (m+1)/2\]
		\[n+1 = m+1\]
		\[n = m\]
		Both cases hold, therefore $f(n)$ is one-to-one.\\ \\
		The next step is to prove $f(n)$ is onto.  \\
		Then we must prove 2 cases: \\
		Case 1: $n$ is even \\
		Let $\forall m \in \Integers^- \cup \{0\},\ \exists n \in \Naturals \ f(n) = m$.
		\[f(n) = m \]
		\[-(n/2) = m\]
		\[n/2 = -m \]
		\[n = -2m\]
		By definition of even numbers, every $n$ can be reached. \\
		\\
		Case 2: $n$ is odd \\
		Let $\forall m \in \Integers^+ , \exists n \in \Naturals \ f(n) = m$.
		\[f(n) = m\]
		\[(n+1)/2 = m\]
		\[n+1 = 2m\]
		\[n = 2m +1\]
		By definition of odd numbers, every $n$ can be reached. \\
		Therefore $\Integers^-\cup \{0\}\cup \Integers^+ = \Integers$ and all value in the codomain can be reached. \\
		\\
		Both cases are satisfied, so $f(x)$ is a bijective function. By definition of equinumerous, $\Naturals$ and $\Integers$ are equinumerous. 
	\item
		$f(x)$ is one-to-one, but not onto for $f(x)=x^2$.\\
		$f(x)$ is onto, but is not one-to-one for\\
		\[f(x)=
			\begin{cases}
			\sqrt{x} & \text{if $x\geq0$}\\
			-\sqrt{x} & \text{if $x<0$}\\
			\end{cases} 
		\]
	\item
		To prove $R$ is an equivalence relation, we must prove it is reflexive, symmetric, and transitive.\\
		Reflexive: $\forall a\in\Naturals, (a,a)\in R = (a-a)\ mod\ 3 = 0\ mod\ 3 = 0$.\\
		\\
		Symmetric: $\forall a, b\in\Naturals$, let $(a-b)\in R$. Then $(a-b)\ mod\ 3$ or $\exists x\in \Naturals\ , a-b=3x$. Note the equation can be rewritten as $b-a=-3x$. $-3x\ mod\ 3=0$ therefore $b-a\ mod\ 3=0$ and $(b,a)\in R$.\\
		\\
		Transitive: Suppose $a,b,c\in \Naturals$ and $(a,b)\in R$ and $(b,c)\in R$. Let $a-b=3x$ and $b-c=3y$. Then $a-c=3x+3y$. Note $3x+3y=3(x+y)$. Therefore $(a,c)\in R$.\\
		\\
		Relations $R$ has three equivalence classes:
		\begin{itemize}
			\item $\{0, 3, 6, ...\}$
			\item $\{1, 4, 7, ...\}$
			\item $\{2, 5, 8, ...\}$
		\end{itemize}
	\item
		For this proof, we will be inducting $n$ from the formula.\\
		Base case: Note $\Sigma_{i=1}^1 i^2=(2*1+1)(1+1)1/6 = 1$.\\
		\\
		Inductive Step: $\Sigma_{i=1}^{n+1} i^2 = \Sigma_{i=1}^n i^2 + (n+1)^2$.
		The inductive hypothesis is $\Sigma_{i=1}^n i^2 = (2n+1)(n+1)n/6$.	Using the I.H.,
		\[\Sigma_{i=1}^{n+1} i^2 =\]
		\[(2n+1)(n+1)n/6 + (n+1)^2 =\]
		\[\frac{n^3}{3}+\frac{3n^2}{2}+\frac{13n}{6}+1 =\]
		\[(2n^3+9n^2+13n+6)\div6=\]
		\[(2n^2+7n+6)(n+1)\div6=\]
		\[(2(n+1)+1)((n+1)+1)(n+1)\div6\]
		It follows by inductions that $\Sigma_{i=1}^{n+1} i^2=(2*1+1)(1+1)1/6 = 1$
	\end{enumerate}
\end{document} 