\documentclass{article}

\usepackage{amsmath}
\usepackage{amssymb}
\usepackage{graphicx}
\usepackage{epsfig}

% epsf figures (inline or centered)
\newcommand{\FIG}[1]{\begin{center}
  \mbox{\epsfclipoff\epsffile{#1.eps}}
  \end{center}}
\def\epsfsize#1#2{0.5#1}	      % change 1.0 to any zoom factor you want

% numeric sets
\newcommand{\Reals}{\mathbb{R}}      % real numbers
\newcommand{\Naturals}{\mathbb{N}}   % natural numbers
\newcommand{\Integers}{\mathbb{Z}}   % integer numbers
\newcommand{\Rationals}{\mathbb{Q}}  % rational numbers
\newcommand{\Complexes}{\mathbb{C}}  % complex numbers
 
 \begin{document}
 	\title{COM S 331\large \\Homework 3}
	\author{Christian Shinkle}
	\maketitle
	\begin{enumerate}
		\item Proof: First, we show that any finite language can be represented by a regular expression.  We build the regular expression by taking the union of all possible words that can be generated by the language.\[ n\in\Naturals,\  L(\Sigma)=\{a_0,a_1,...a_n\} : R= a_0+ a_1+ ...+ a_n.\] 
		Because $\Sigma$ is finite, every possible word made by $\Sigma$ is represented in the regular expression R. By definition of Regular Languages (from section 2.7), L must be regular. 
		\\
		\\
		Now that L is proven to be regular, using the theorem from section 2.7, L must also be accepted by a finite automaton. 
		\item .\FIG{hw3DFA1}
		\newpage
		\item .\FIG{hw3DFA2}
	\end{enumerate}
 \end{document}